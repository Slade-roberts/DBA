\chapter{Task 5a: Stored Procedures}

This chapter includes the stored procedures for Task 5a, with code excerpts and screenshots. All stored procedures have been implemented and tested in Azure SQL Database.

\section{Stored Procedure 1: sp\_InsertVisitorWithEnrollments}

This stored procedure inserts a new person record, creates their visitor role, and optionally enrolls them in multiple programs using a CSV-formatted string.

\begin{lstlisting}[label={lst5a:sp_insertvisitorwithenrollments}]
IF OBJECT_ID('sp_InsertVisitorWithEnrollments','P') IS NOT NULL
	DROP PROCEDURE sp_InsertVisitorWithEnrollments;
GO
-- ===========================================================================
-- sp_InsertVisitorWithEnrollments
-- Purpose: insert a person and their visitor role, and optionally enroll 
--          them in programs.
--
-- Parameters (explained):
--  @PersonID    : identifier for the person (NVARCHAR(50)). I used text 
--                 IDs to make it easy to provide meaningful IDs during 
--                 tests (instead of opaque integers).
--  @FirstName,@MiddleInitial,@LastName,@DOB,@Gender, etc.: basic person 
--                 attributes used to populate the Person table. DOB should 
--                 be in YYYY-MM-DD format.
--  @Newsletter  : 0/1 flag for mailing-list opt-in.
--  @PassID      : optional link to ParkPass; NULL if the visitor has no pass.
--  @Enrollments : CSV like '12:2025-06-01,15:2025-06-05' where each piece 
--                 is ProgramID:VisitDate. The procedure parses this string 
--                 server-side and inserts Enrollment rows.
-- Behavior notes:
--  - The procedure uses an explicit transaction so Person/Visitor/Enrollment 
--    are inserted together; on error the transaction is rolled back and an 
--    error is raised for the client.
-- ===========================================================================
CREATE PROCEDURE sp_InsertVisitorWithEnrollments
	@PersonID NVARCHAR(50),
	@FirstName NVARCHAR(100),
	@MiddleInitial NCHAR(1) = NULL,
	@LastName NVARCHAR(100),
	@DOB DATE,
	@Gender NVARCHAR(20) = NULL,
	@Street NVARCHAR(200) = NULL,
	@City NVARCHAR(100) = NULL,
	@State NVARCHAR(50) = NULL,
	@PostalCode NVARCHAR(20) = NULL,
	@Newsletter BIT = 0,
	@PassID INT = NULL,
	@Enrollments NVARCHAR(MAX) = NULL
AS
BEGIN
	SET NOCOUNT ON;
	BEGIN TRY
		BEGIN TRANSACTION;
		-- Only insert Person if they do not already exist (idempotent)
		IF NOT EXISTS (SELECT 1 FROM Person WHERE PersonID = @PersonID)
		BEGIN
			INSERT INTO Person(PersonID, FirstName, MiddleInitial, LastName, 
			                   DOB, Gender, Street, City, State, PostalCode, 
			                   NewsletterSubscribed)
			VALUES(@PersonID, @FirstName, @MiddleInitial, @LastName, @DOB, 
			       @Gender, @Street, @City, @State, @PostalCode, @Newsletter);
		END

		-- Only insert Visitor role if not already present
		IF NOT EXISTS (SELECT 1 FROM Visitor WHERE VisitorID = @PersonID)
		BEGIN
			INSERT INTO Visitor(VisitorID, PassID) 
			VALUES(@PersonID, @PassID);
		END

		IF (@Enrollments IS NOT NULL AND LTRIM(RTRIM(@Enrollments)) <> '')
		BEGIN
			DECLARE @xml XML = CAST('<x>' + REPLACE(@Enrollments, ',', '</x><x>') 
			                        + '</x>' AS XML);
			DECLARE @e TABLE(ProgramID INT, VisitDate DATE);

			INSERT INTO @e(ProgramID, VisitDate)
			SELECT
				TRY_CAST(LEFT(T.value('.', 'nvarchar(max)'), 
				         CHARINDEX(':', T.value('.', 'nvarchar(max)')) - 1) 
				         AS INT),
				TRY_CAST(SUBSTRING(T.value('.', 'nvarchar(max)'), 
				         CHARINDEX(':', T.value('.', 'nvarchar(max)')) + 1, 50) 
				         AS DATE)
			FROM @xml.nodes('/x') AS X(T);

			-- Insert enrollments but avoid duplicates
			INSERT INTO Enrollment(VisitorID, ProgramID, VisitDate)
			SELECT @PersonID, e.ProgramID, e.VisitDate
			FROM @e e
			WHERE NOT EXISTS (
				SELECT 1 FROM Enrollment en 
				WHERE en.VisitorID = @PersonID 
				  AND en.ProgramID = e.ProgramID 
				  AND en.VisitDate = e.VisitDate
			);
		END

		COMMIT TRANSACTION;
	END TRY
	BEGIN CATCH
		IF @@TRANCOUNT > 0
			ROLLBACK TRANSACTION;
		DECLARE @ErrorMessage NVARCHAR(4000) = ERROR_MESSAGE();
		RAISERROR('Error inserting visitor: %s', 16, 1, @ErrorMessage);
		RETURN;
	END CATCH
END
GO
\end{lstlisting}

\begin{figure}[H]
\centering
\includegraphics[width=\textwidth]{screenshots/task5a_sp1.png}
\caption{Screenshot of sp\_InsertVisitorWithEnrollments code (part 1)}
\end{figure}

\begin{figure}[H]
\centering
\includegraphics[width=\textwidth]{screenshots/task5a_sp1.2.png}
\caption{Screenshot of sp\_InsertVisitorWithEnrollments code (part 2)}
\end{figure}

\section{Stored Procedure 2: sp\_InsertRanger}

This stored procedure inserts a new ranger into the system, creating a minimal Person record if one doesn't exist.

\begin{lstlisting}[label={lst5a:sp_insertranger}]
IF OBJECT_ID('sp_InsertRanger','P') IS NOT NULL
	DROP PROCEDURE sp_InsertRanger;
GO
CREATE PROCEDURE sp_InsertRanger
	@RangerID NVARCHAR(50),
	@StartDate DATE,
	@Status NVARCHAR(20)
AS
BEGIN
	SET NOCOUNT ON;
	BEGIN TRY
		-- Insert Person record if it doesn't exist
		IF NOT EXISTS (SELECT 1 FROM Person WHERE PersonID = @RangerID)
		BEGIN
			INSERT INTO Person(PersonID, FirstName, LastName, DOB) 
			VALUES(@RangerID, 'Unknown','Unknown','1900-01-01');
		END

		-- Insert Ranger record if it doesn't exist
		IF NOT EXISTS (SELECT 1 FROM Ranger WHERE RangerID = @RangerID)
		BEGIN
			INSERT INTO Ranger(RangerID, StartDate, Status) 
			VALUES(@RangerID, @StartDate, @Status);
		END
	END TRY
	BEGIN CATCH
		THROW;
	END CATCH
END
GO
\end{lstlisting}

\begin{figure}[H]
\centering
\includegraphics[width=\textwidth]{screenshots/task5a_sp2.png}
\caption{Screenshot of sp\_InsertRanger code}
\end{figure}

\section{Stored Procedure 3: sp\_InsertRangerTeamWithLeader}

This stored procedure creates a new ranger team and assigns a leader to that team.

\begin{lstlisting}[label={lst5a:sp_insertrangerteamwithleader}]
IF OBJECT_ID('sp_InsertRangerTeamWithLeader','P') IS NOT NULL
	DROP PROCEDURE sp_InsertRangerTeamWithLeader;
GO
CREATE PROCEDURE sp_InsertRangerTeamWithLeader
	@FocusArea NVARCHAR(200),
	@FormationDate DATE = NULL,
	@LeaderRangerID NVARCHAR(50)
AS
BEGIN
	SET NOCOUNT ON;
	-- Check if the leader ranger is already assigned to a team
	IF NOT EXISTS (SELECT 1 FROM RangerTeamAssignment 
	               WHERE RangerID = @LeaderRangerID)
	BEGIN
		DECLARE @TeamID INT;
		INSERT INTO RangerTeam(FocusArea, FormationDate) 
		VALUES(@FocusArea, @FormationDate);
		SET @TeamID = SCOPE_IDENTITY();
		INSERT INTO RangerTeamAssignment(TeamID, RangerID, 
		                                 AssignmentStartDate, Status, IsLeader) 
		VALUES(@TeamID, @LeaderRangerID, GETDATE(), 'active', 1);
	END
END
GO
\end{lstlisting}

\begin{figure}[H]
\centering
\includegraphics[width=\textwidth]{screenshots/task5a_sp3.png}
\caption{Screenshot of sp\_InsertRangerTeamWithLeader code}
\end{figure}

\section{Stored Procedure 4: sp\_InsertDonation}

This stored procedure inserts a donation record and the appropriate payment details (check or credit card).

\begin{lstlisting}[label={lst5a:sp_insertdonation}]
IF OBJECT_ID('sp_InsertDonation','P') IS NOT NULL
	DROP PROCEDURE sp_InsertDonation;
GO
CREATE PROCEDURE sp_InsertDonation
	@DonorID NVARCHAR(50),
	@DonationDate DATE,
	@Amount DECIMAL(14,2),
	@CampaignName NVARCHAR(200) = NULL,
	@PaymentType NVARCHAR(20),
	@CheckNumber NVARCHAR(100) = NULL,
	@CardType NVARCHAR(50) = NULL,
	@LastFour NCHAR(4) = NULL,
	@ExpDate DATE = NULL
AS
BEGIN
	SET NOCOUNT ON;
	DECLARE @DonationID INT;
	INSERT INTO Donation(DonorID, DonationDate, Amount, CampaignName, PaymentType)
	VALUES(@DonorID, @DonationDate, @Amount, @CampaignName, @PaymentType);
	SET @DonationID = SCOPE_IDENTITY();
	IF @PaymentType = 'CHECK'
	BEGIN
		-- Note: For CHECK payments the check number is stored in a 
		-- separate CheckPayment table; CheckPayment.DonationID is a 
		-- 1:1 PK referencing Donation.
		INSERT INTO CheckPayment(DonationID, CheckNumber) 
		VALUES(@DonationID, @CheckNumber);
	END
	ELSE IF @PaymentType = 'CARD'
	BEGIN
		-- Note: For CARD payments we only store non-sensitive metadata 
		-- (card type, last four, expiration). We intentionally do NOT 
		-- store full PANs or CVV values for security.
		INSERT INTO CreditCardPayment(DonationID, CardType, LastFourDigits, 
		                               ExpirationDate) 
		VALUES(@DonationID, @CardType, @LastFour, @ExpDate);
	END
END
GO
\end{lstlisting}

\begin{figure}[H]
\centering
\includegraphics[width=\textwidth]{screenshots/task5a_sp4.png}
\caption{Screenshot of sp\_InsertDonation code}
\end{figure}

\section{Stored Procedure 5: sp\_InsertResearcher}

This stored procedure inserts a new researcher into the system.

\begin{lstlisting}[label={lst5a:sp_insertresearcher}]
IF OBJECT_ID('sp_InsertResearcher','P') IS NOT NULL
	DROP PROCEDURE sp_InsertResearcher;
GO
CREATE PROCEDURE sp_InsertResearcher
	@ResearcherID NVARCHAR(50),
	@Field NVARCHAR(200),
	@HireDate DATE,
	@Salary DECIMAL(12,2)
AS
BEGIN
	SET NOCOUNT ON;
	BEGIN TRY
		-- Insert Person record if it doesn't exist
		IF NOT EXISTS (SELECT 1 FROM Person WHERE PersonID = @ResearcherID)
		BEGIN
			INSERT INTO Person(PersonID, FirstName, LastName, DOB) 
			VALUES(@ResearcherID, 'Unknown','Unknown','1900-01-01');
		END

		-- Insert Researcher record if it doesn't exist
		IF NOT EXISTS (SELECT 1 FROM Researcher 
		               WHERE ResearcherID = @ResearcherID)
		BEGIN
			INSERT INTO Researcher(ResearcherID, Field, HireDate, Salary) 
			VALUES(@ResearcherID, @Field, @HireDate, @Salary);
		END
	END TRY
	BEGIN CATCH
		THROW;
	END CATCH
END
GO
\end{lstlisting}

\begin{figure}[H]
\centering
\includegraphics[width=\textwidth]{screenshots/task5a_sp5.png}
\caption{Screenshot of sp\_InsertResearcher code}
\end{figure}

\section{Stored Procedure 6: sp\_InsertReport}

This stored procedure inserts a new research report.

\begin{lstlisting}[label={lst5a:sp_insertreport}]
IF OBJECT_ID('sp_InsertReport','P') IS NOT NULL
	DROP PROCEDURE sp_InsertReport;
GO
CREATE PROCEDURE sp_InsertReport
	@TeamID INT,
	@ResearcherID NVARCHAR(50),
	@ReportDate DATE,
	@Summary NVARCHAR(MAX)
AS
BEGIN
	SET NOCOUNT ON;
	INSERT INTO Report(TeamID, ResearcherID, ReportDate, Summary) 
	VALUES(@TeamID, @ResearcherID, @ReportDate, @Summary);
END
GO
\end{lstlisting}

\begin{figure}[H]
\centering
\includegraphics[width=\textwidth]{screenshots/task5a_sp6.png}
\caption{Screenshot of sp\_InsertReport code}
\end{figure}

\section{Stored Procedure 7: sp\_InsertProgram}

This stored procedure inserts a new park program.

\begin{lstlisting}[label={lst5a:sp_insertprogram}]
IF OBJECT_ID('sp_InsertProgram','P') IS NOT NULL
	DROP PROCEDURE sp_InsertProgram;
GO
CREATE PROCEDURE sp_InsertProgram
	@ParkName NVARCHAR(200),
	@ProgramName NVARCHAR(200),
	@Type NVARCHAR(100),
	@StartDate DATE,
	@DurationHours INT
AS
BEGIN
	SET NOCOUNT ON;
	INSERT INTO Program(ParkName, ProgramName, Type, StartDate, DurationHours) 
	VALUES(@ParkName, @ProgramName, @Type, @StartDate, @DurationHours);
END
GO
\end{lstlisting}

\begin{figure}[H]
\centering
\includegraphics[width=\textwidth]{screenshots/task5a_sp7.png}
\caption{Screenshot of sp\_InsertProgram code}
\end{figure}

\section{Stored Procedure 8: sp\_GetEmergencyContacts}

This stored procedure retrieves all emergency contacts for a given person.

\begin{lstlisting}[label={lst5a:sp_getemergencycontacts}]
IF OBJECT_ID('sp_GetEmergencyContacts','P') IS NOT NULL
	DROP PROCEDURE sp_GetEmergencyContacts;
GO
CREATE PROCEDURE sp_GetEmergencyContacts
	@PersonID NVARCHAR(50)
AS
BEGIN
	SET NOCOUNT ON;
	SELECT ContactName, Relationship, PhoneNumber 
	FROM EmergencyContact 
	WHERE PersonID = @PersonID;
END
GO
\end{lstlisting}

\begin{figure}[H]
\centering
\includegraphics[width=\textwidth]{screenshots/task5a_sp8.png}
\caption{Screenshot of sp\_GetEmergencyContacts code}
\end{figure}

\section{Stored Procedure 9: sp\_GetVisitorsByProgram}

This stored procedure retrieves all visitors enrolled in a specific program.

\begin{lstlisting}[label={lst5a:sp_getvisitorsbyprogram}]
IF OBJECT_ID('sp_GetVisitorsByProgram','P') IS NOT NULL
	DROP PROCEDURE sp_GetVisitorsByProgram;
GO
CREATE PROCEDURE sp_GetVisitorsByProgram
	@ProgramID INT
AS
BEGIN
	SET NOCOUNT ON;
	SELECT p.PersonID, p.FirstName, p.LastName, e.VisitDate, 
	       e.AccessibilityNeeds
	FROM Enrollment e
	JOIN Visitor v ON e.VisitorID = v.VisitorID
	JOIN Person p ON v.VisitorID = p.PersonID
	WHERE e.ProgramID = @ProgramID;
END
GO
\end{lstlisting}

\begin{figure}[H]
\centering
\includegraphics[width=\textwidth]{screenshots/task5a_sp9.png}
\caption{Screenshot of sp\_GetVisitorsByProgram code}
\end{figure}

\section{Stored Procedure 10: sp\_GetProgramsAfterDate}

This stored procedure retrieves all programs at a park that start after a given date.

\begin{lstlisting}[label={lst5a:sp_getprogramsafterdate}]
IF OBJECT_ID('sp_GetProgramsAfterDate','P') IS NOT NULL
	DROP PROCEDURE sp_GetProgramsAfterDate;
GO
CREATE PROCEDURE sp_GetProgramsAfterDate
	@ParkName NVARCHAR(200),
	@StartAfter DATE
AS
BEGIN
	SET NOCOUNT ON;
	SELECT * 
	FROM Program 
	WHERE ParkName = @ParkName AND StartDate > @StartAfter 
	ORDER BY StartDate;
END
GO
\end{lstlisting}

\begin{figure}[H]
\centering
\includegraphics[width=\textwidth]{screenshots/task5a_sp10.png}
\caption{Screenshot of sp\_GetProgramsAfterDate code}
\end{figure}

\section{Stored Procedure 11: sp\_AnonymousDonationStatsByMonth}

This stored procedure calculates donation statistics for anonymous donors for a given month.

\begin{lstlisting}[label={lst5a:sp_anonymousdonationstatsbymonth}]
IF OBJECT_ID('sp_AnonymousDonationStatsByMonth','P') IS NOT NULL
	DROP PROCEDURE sp_AnonymousDonationStatsByMonth;
GO
CREATE PROCEDURE sp_AnonymousDonationStatsByMonth
	@Year INT,
	@Month INT
AS
BEGIN
	SET NOCOUNT ON;
	SELECT d.DonorID, SUM(d.Amount) AS TotalAmount, 
	       AVG(d.Amount) AS AvgAmount, COUNT(*) AS NumDonations
	FROM Donation d
	JOIN Donor o ON d.DonorID = o.DonorID
	WHERE o.IsAnonymous = 1 
	  AND YEAR(d.DonationDate) = @Year 
	  AND MONTH(d.DonationDate) = @Month
	GROUP BY d.DonorID
	ORDER BY SUM(d.Amount) DESC;
END
GO
\end{lstlisting}

\begin{figure}[H]
\centering
\includegraphics[width=\textwidth]{screenshots/task5a_sp11.png}
\caption{Screenshot of sp\_AnonymousDonationStatsByMonth code}
\end{figure}

\section{Stored Procedure 12: sp\_GetRangersInTeam}

This stored procedure retrieves all rangers in a specific team with their details.

\begin{lstlisting}[label={lst5a:sp_getrangersinteam}]
IF OBJECT_ID('sp_GetRangersInTeam','P') IS NOT NULL
	DROP PROCEDURE sp_GetRangersInTeam;
GO
CREATE PROCEDURE sp_GetRangersInTeam
	@TeamID INT
AS
BEGIN
	SET NOCOUNT ON;
	SELECT r.RangerID, p.FirstName, p.LastName, 
	       DATEDIFF(YEAR, r.StartDate, GETDATE()) AS YearsOfService, 
	       rta.IsLeader, rc.Certification
	FROM RangerTeamAssignment rta
	JOIN Ranger r ON rta.RangerID = r.RangerID
	JOIN Person p ON r.RangerID = p.PersonID
	LEFT JOIN RangerCertification rc ON r.RangerID = rc.RangerID
	WHERE rta.TeamID = @TeamID
	ORDER BY rta.IsLeader DESC, p.LastName;
END
GO
\end{lstlisting}

\begin{figure}[H]
\centering
\includegraphics[width=\textwidth]{screenshots/task5a_sp12.png}
\caption{Screenshot of sp\_GetRangersInTeam code}
\end{figure}

\section{Stored Procedure 13: sp\_GetAllIndividuals}

This stored procedure retrieves all individuals in the system.

\begin{lstlisting}[label={lst5a:sp_getallindividuals}]
IF OBJECT_ID('sp_GetAllIndividuals','P') IS NOT NULL
	DROP PROCEDURE sp_GetAllIndividuals;
GO
CREATE PROCEDURE sp_GetAllIndividuals
AS
BEGIN
	SET NOCOUNT ON;
	SELECT p.PersonID, p.FirstName, p.MiddleInitial, p.LastName, p.DOB, 
	       p.Gender, p.Street, p.City, p.State, p.PostalCode, 
	       p.NewsletterSubscribed
	FROM Person p;
END
GO
\end{lstlisting}

\begin{figure}[H]
\centering
\includegraphics[width=\textwidth]{screenshots/task5a_sp13.png}
\caption{Screenshot of sp\_GetAllIndividuals code}
\end{figure}

\section{Stored Procedure 14: sp\_IncreaseSalary}

This stored procedure increases salaries by 3\% for researchers overseeing multiple teams.

\begin{lstlisting}[label={lst5a:sp_increasesalary}]
IF OBJECT_ID('sp_IncreaseSalary','P') IS NOT NULL
	DROP PROCEDURE sp_IncreaseSalary;
GO
CREATE PROCEDURE sp_IncreaseSalary
AS
BEGIN
	SET NOCOUNT ON;
	UPDATE r
	SET Salary = Salary * 1.03
	FROM Researcher r
	WHERE r.ResearcherID IN (
		SELECT ResearcherID 
		FROM RangerTeamOverseen 
		GROUP BY ResearcherID 
		HAVING COUNT(*) > 1
	);
END
GO
\end{lstlisting}

\begin{figure}[H]
\centering
\includegraphics[width=\textwidth]{screenshots/task5a_sp14.png}
\caption{Screenshot of sp\_IncreaseSalary code}
\end{figure}

\section{Stored Procedure 15: sp\_DeleteInactiveVisitors}

This stored procedure deletes visitors who have no enrollments and either no pass or an expired pass.

\begin{lstlisting}[label={lst5a:sp_deleteinactivevisitors}]
IF OBJECT_ID('sp_DeleteInactiveVisitors','P') IS NOT NULL
	DROP PROCEDURE sp_DeleteInactiveVisitors;
GO
CREATE PROCEDURE sp_DeleteInactiveVisitors
AS
BEGIN
	SET NOCOUNT ON;
	DELETE v
	FROM Visitor v
	LEFT JOIN Enrollment e ON v.VisitorID = e.VisitorID
	LEFT JOIN ParkPass pp ON v.PassID = pp.PassID
	WHERE e.VisitorID IS NULL 
	  AND (pp.PassID IS NULL OR pp.ExpirationDate < GETDATE());
END
GO
\end{lstlisting}

\begin{figure}[H]
\centering
\includegraphics[width=\textwidth]{screenshots/task5a_sp15.png}
\caption{Screenshot of sp\_DeleteInactiveVisitors code}
\end{figure}

\section{Test Data and Procedure Calls}

The following SQL script demonstrates testing of the stored procedures with sample data.

\begin{lstlisting}[label={lst5a:test_data_and_calls}]
-- Consolidated additional test data and calls from extras
-- test_data_and_calls.sql
SET NOCOUNT ON;
PRINT '=== START TEST DATA AND PROCEDURE CALLS ===';

-- 1) Create a program and capture its ProgramID
PRINT 'Ensuring NationalPark "Grand Park" exists...';
IF NOT EXISTS (SELECT 1 FROM NationalPark WHERE ParkName = 'Grand Park')
BEGIN
    INSERT INTO NationalPark(ParkName, Street, City, State, PostalCode, 
                             EstablishmentDate, Capacity)
    VALUES('Grand Park','1 Park Ave','Parkville','PS','12345',
           '2000-01-01',1000);
END

PRINT 'Inserting Program...';
EXEC sp_InsertProgram @ParkName='Grand Park', @ProgramName='NatureWalk', 
                      @Type='Education', @StartDate='2025-11-01', 
                      @DurationHours=2;

DECLARE @ProgramID INT;
SELECT @ProgramID = MAX(ProgramID) 
FROM Program 
WHERE ParkName='Grand Park' AND ProgramName='NatureWalk';
PRINT 'ProgramID=' + COALESCE(CAST(@ProgramID AS NVARCHAR(20)),'(null)');

-- 2) Insert a visitor and enroll them in the just-created program
PRINT 'Inserting Visitor with enrollment (idempotent)...';
DECLARE @VisitorEnrollments NVARCHAR(MAX) = NULL;
IF @ProgramID IS NOT NULL
    SET @VisitorEnrollments = CAST(@ProgramID AS NVARCHAR(10)) + ':2025-11-09';

IF NOT EXISTS (SELECT 1 FROM Person WHERE PersonID = 'vis100')
BEGIN
    EXEC sp_InsertVisitorWithEnrollments
        @PersonID='vis100',
        @FirstName='Sam',
        @MiddleInitial=NULL,
        @LastName='Visitor',
        @DOB='1990-05-20',
        @Gender='Non-binary',
        @Street='100 Trail Rd',
        @City='Parkville',
        @State='PS',
        @PostalCode='12345',
        @Newsletter=1,
        @PassID=NULL,
        @Enrollments = @VisitorEnrollments;
END
ELSE
BEGIN
    PRINT 'Visitor vis100 already exists; skipping.';
END

-- 3) Insert a donor and a donation (card)
PRINT 'Inserting Donor and Donation (idempotent)...';
IF NOT EXISTS (SELECT 1 FROM Person WHERE PersonID = 'donor100')
    INSERT INTO Person(PersonID, FirstName, LastName, DOB) 
    VALUES('donor100','Alice','Smith','1980-01-01');
IF NOT EXISTS (SELECT 1 FROM Donor WHERE DonorID = 'donor100')
    INSERT INTO Donor(DonorID, IsAnonymous) VALUES('donor100',0);
EXEC sp_InsertDonation @DonorID='donor100', @DonationDate='2025-11-01', 
                       @Amount=150.00, @CampaignName='FallCampaign', 
                       @PaymentType='CARD', @CheckNumber=NULL, 
                       @CardType='VISA', @LastFour='4242', 
                       @ExpDate='2026-12-31';

-- 4) Insert an anonymous donor and donation
PRINT 'Inserting Anonymous Donor and Donation (idempotent)...';
IF NOT EXISTS (SELECT 1 FROM Person WHERE PersonID = 'anon1')
    INSERT INTO Person(PersonID, FirstName, LastName, DOB) 
    VALUES('anon1','Anonymous','Donor','1970-01-01');
IF NOT EXISTS (SELECT 1 FROM Donor WHERE DonorID = 'anon1')
    INSERT INTO Donor(DonorID, IsAnonymous) VALUES('anon1',1);
EXEC sp_InsertDonation @DonorID='anon1', @DonationDate='2025-11-05', 
                       @Amount=75.00, @CampaignName='FallCampaign', 
                       @PaymentType='CHECK', @CheckNumber='1001', 
                       @CardType=NULL, @LastFour=NULL, @ExpDate=NULL;

-- 5) Call retrieval procedures to verify outputs
PRINT 'Retrieving visitors enrolled in program...';
IF @ProgramID IS NOT NULL
    EXEC sp_GetVisitorsByProgram @ProgramID=@ProgramID;

PRINT 'Retrieving all individuals...';
EXEC sp_GetAllIndividuals;

PRINT 'Anonymous donation stats for Nov 2025...';
EXEC sp_AnonymousDonationStatsByMonth @Year=2025, @Month=11;

-- 6) Verify procedures are present
PRINT 'Verifying rangers-in-team (may be empty)...';
EXEC sp_GetRangersInTeam @TeamID=1;

PRINT '=== END TEST DATA AND PROCEDURE CALLS ===';
GO
\end{lstlisting}

\section{Summary}

All 15 stored procedures have been successfully created and tested in Azure SQL Database. The procedures provide comprehensive functionality for:

\begin{itemize}
    \item Inserting new records (visitors, rangers, teams, donations, researchers, reports, programs)
    \item Retrieving data (emergency contacts, visitors by program, programs by date, rangers in team, all individuals, anonymous donation statistics)
    \item Updating data (increasing salaries for multi-team researchers)
    \item Deleting data (removing inactive visitors)
\end{itemize}

Each procedure includes appropriate error handling, parameter validation, and transactional integrity where needed. The procedures are designed to be called from the Java application developed in Task 5b.