\chapter{Task 5b: Java Application}

This chapter includes the Java application code developed for Task 5b. The application provides a menu-driven interface for interacting with all stored procedures created in Task 5a, plus import/export functionality.

\section{Application Overview}

The Java application \texttt{Roberts\_Slade\_IP\_Task5b.java} is a single-file console application that:
\begin{itemize}
    \item Connects to Azure SQL Database using JDBC
    \item Provides an interactive menu with 18 options
    \item Calls all 15 stored procedures from Task 5a
    \item Supports importing team data from CSV files
    \item Exports mailing list and all table data to CSV files
    \item Includes comprehensive error handling
    \item Automatically exports all tables after modifying operations
\end{itemize}

\section{Imports and Class Declaration}

\begin{lstlisting}[label={lst5b:imports}]
import java.io.BufferedReader;
import java.io.BufferedWriter;
import java.io.FileInputStream;
import java.io.FileOutputStream;
import java.io.FileReader;
import java.io.IOException;
import java.io.OutputStreamWriter;
import java.io.Writer;
import java.nio.charset.StandardCharsets;
import java.nio.file.Files;
import java.nio.file.Path;
import java.nio.file.Paths;
import java.sql.CallableStatement;
import java.sql.Connection;
import java.sql.Date;
import java.sql.DriverManager;
import java.sql.PreparedStatement;
import java.sql.ResultSet;
import java.sql.ResultSetMetaData;
import java.sql.SQLException;
import java.sql.Statement;
import java.sql.Types;
import java.text.SimpleDateFormat;
import java.util.Arrays;
import java.util.List;
import java.util.Properties;
import java.util.Scanner;

/**
 * Roberts_Slade_IP_Task5b
 * Single-file Java menu client for Task 5 stored procedures.
 * Author: Slade Roberts
 * Date: 2025-11-09
 */
public class Roberts_Slade_IP_Task5b {
	private static Connection conn;
	private static final Scanner sc = new Scanner(System.in);

	private static final List<String> TABLES = Arrays.asList(
		"Person", "PersonPhone", "PersonEmail", "EmergencyContact",
		"ParkPass", "Visitor", "Ranger", "RangerCertification",
		"RangerTeam", "RangerTeamAssignment", "Mentorship",
		"Researcher", "RangerTeamOverseen", "NationalPark",
		"Program", "Enrollment", "Project", "Donor",
		"Donation", "CheckPayment", "CreditCardPayment", "Report"
	);

	private static final Path EXPORT_DIR = Paths.get("../table_results");
	private static final SimpleDateFormat TIMESTAMP_FORMAT = 
	    new SimpleDateFormat("yyyy-MM-dd HH:mm:ss");
\end{lstlisting}

\begin{figure}[H]
\centering
\includegraphics[width=\textwidth]{screenshots/task5b_imports.png}
\caption{Screenshot of imports and class declaration}
\end{figure}

\section{Main Method}

The main method establishes database connection and runs the main menu loop.

\begin{lstlisting}[label={lst5b:main}]
public static void main(String[] args) throws Exception {
	Properties props = new Properties();
	try (FileInputStream fis = new FileInputStream("db.properties")) {
		props.load(fis);
	} catch (Exception e) {
		System.out.println("Please create db.properties in project root.");
		System.out.println(e.getMessage());
		return;
	}

	String url = props.getProperty("jdbc.url");
	String user = props.getProperty("jdbc.user");
	String pass = props.getProperty("jdbc.password");
	conn = DriverManager.getConnection(url, user, pass);
	System.out.println("Connected to database.");

	while (true) {
		printMenu();
		String choice = sc.nextLine().trim();
		try {
			int opt = Integer.parseInt(choice);
			if (opt == 18) break;
			handleOption(opt);
		} catch (NumberFormatException ex) {
			System.out.println("Invalid option. Enter a number.");
		}
	}

	conn.close();
	System.out.println("Goodbye.");
}
\end{lstlisting}

\begin{figure}[H]
\centering
\includegraphics[width=\textwidth]{screenshots/task5b_main.png}
\caption{Screenshot of main method}
\end{figure}

\section{Print Menu Method}

Displays the interactive menu with all 18 options.

\begin{lstlisting}[label={lst5b:printmenu}]
private static void printMenu() {
	System.out.println("\nWELCOME TO THE NATIONAL PARK SERVICE SYSTEM DATABASE");
	System.out.println("1) Insert a new visitor and enroll in programs");
	System.out.println("2) Insert a new ranger");
	System.out.println("3) Insert a new ranger team and set its leader");
	System.out.println("4) Insert a new donation");
	System.out.println("5) Insert a new researcher");
	System.out.println("6) Insert a report by a ranger team to a researcher");
	System.out.println("7) Insert a new park program");
	System.out.println("8) Retrieve emergency contacts for a person");
	System.out.println("9) Retrieve visitors enrolled in a program");
	System.out.println("10) Retrieve park programs after a date");
	System.out.println("11) Anonymous donation stats for a month");
	System.out.println("12) List rangers in a team");
	System.out.println("13) List all individuals");
	System.out.println("14) Increase salary for multi-team researchers");
	System.out.println("15) Delete inactive visitors");
	System.out.println("16) Import teams from file (CSV)");
	System.out.println("17) Export mailing list to file (CSV)");
	System.out.println("18) Quit");
	System.out.print("Choose option: ");
}
\end{lstlisting}

\begin{figure}[H]
\centering
\includegraphics[width=\textwidth]{screenshots/task5b_printmenu.png}
\caption{Screenshot of printMenu method}
\end{figure}

\section{Handle Option Method}

Routes menu selections to appropriate handler methods.

\begin{lstlisting}[label={lst5b:handleoption}]
private static void handleOption(int opt) throws SQLException {
	switch (opt) {
		case 1: doInsertVisitor(); exportAllTables(); break;
		case 2: doInsertRanger(); exportAllTables(); break;
		case 3: doInsertTeam(); exportAllTables(); break;
		case 4: doInsertDonation(); exportAllTables(); break;
		case 5: doInsertResearcher(); exportAllTables(); break;
		case 6: doInsertReport(); exportAllTables(); break;
		case 7: doInsertProgram(); exportAllTables(); break;
		case 8: doGetEmergencyContacts(); break;
		case 9: doGetVisitorsByProgram(); break;
		case 10: doGetProgramsAfterDate(); break;
		case 11: doAnonymousDonationStats(); break;
		case 12: doGetRangersInTeam(); break;
		case 13: doGetAllIndividuals(); break;
		case 14: doIncreaseSalary(); exportAllTables(); break;
		case 15: doDeleteInactiveVisitors(); exportAllTables(); break;
		case 16: doImportTeams(); exportAllTables(); break;
		case 17: doExportMailingList(); break;
		default: System.out.println("Option not implemented.");
	}
}
\end{lstlisting}

\begin{figure}[H]
\centering
\includegraphics[width=\textwidth]{screenshots/task5b_handleoption.png}
\caption{Screenshot of handleOption method}
\end{figure}

\section{Insert Methods}

\subsection{doInsertVisitor Method}

Prompts for visitor information and calls \texttt{sp\_InsertVisitorWithEnrollments}.

\begin{lstlisting}[label={lst5b:doinsertvisitor}]
private static void doInsertVisitor() throws SQLException {
	System.out.print("Enter data manually (M) or from file (F): ");
	String choice = sc.nextLine().toUpperCase();
	String id, fn, ln, dob, nl, pass, enroll;
	
	if (choice.equals("F")) {
		System.out.print("Enter file path: ");
		String filePath = sc.nextLine();
		try (BufferedReader br = new BufferedReader(new FileReader(filePath))) {
			id = br.readLine();
			fn = br.readLine();
			ln = br.readLine();
			dob = br.readLine();
			nl = br.readLine();
			pass = br.readLine();
			enroll = br.readLine();
		} catch (IOException e) {
			System.out.println("Error reading file: " + e.getMessage());
			return;
		}
	} else {
		System.out.print("PersonID: "); id = sc.nextLine();
		System.out.print("FirstName: "); fn = sc.nextLine();
		System.out.print("LastName: "); ln = sc.nextLine();
		System.out.print("DOB (YYYY-MM-DD): "); dob = sc.nextLine();
		System.out.print("Newsletter? (0/1): "); nl = sc.nextLine();
		System.out.print("PassID (or blank): "); pass = sc.nextLine();
		System.out.print("Enrollments (ProgramID:YYYY-MM-DD, comma separated) or blank: ");
		enroll = sc.nextLine();
	}

	try (CallableStatement cstmt = conn.prepareCall(
	    "{call sp_InsertVisitorWithEnrollments(?,?,?,?,?,?,?,?,?,?,?,?,?)}")) {
		cstmt.setString(1, id);
		cstmt.setString(2, fn);
		cstmt.setString(3, null);
		cstmt.setString(4, ln);
		cstmt.setDate(5, Date.valueOf(dob));
		cstmt.setString(6, null);
		cstmt.setString(7, null);
		cstmt.setString(8, null);
		cstmt.setString(9, null);
		cstmt.setString(10, null);
		cstmt.setBoolean(11, "1".equals(nl));
		if (pass.isBlank()) 
		    cstmt.setNull(12, Types.INTEGER); 
		else 
		    cstmt.setInt(12, Integer.parseInt(pass));
		cstmt.setString(13, enroll);
		cstmt.execute();
	}
	System.out.println("Inserted visitor and enrollments (if provided).");
}
\end{lstlisting}

\begin{figure}[H]
\centering
\includegraphics[width=\textwidth]{screenshots/task5b_doinsertvisitor.png}
\caption{Screenshot of doInsertVisitor method}
\end{figure}

\subsection{doInsertRanger Method}

\begin{lstlisting}[label={lst5b:doinsertranger}]
private static void doInsertRanger() throws SQLException {
	System.out.print("Enter data manually (M) or from file (F): ");
	String choice = sc.nextLine().toUpperCase();
	String rid, sd, st;
	
	if (choice.equals("F")) {
		System.out.print("Enter file path: ");
		String filePath = sc.nextLine();
		try (BufferedReader br = new BufferedReader(new FileReader(filePath))) {
			rid = br.readLine();
			sd = br.readLine();
			st = br.readLine();
		} catch (IOException e) {
			System.out.println("Error reading file: " + e.getMessage());
			return;
		}
	} else {
		System.out.print("RangerID (PersonID): "); rid = sc.nextLine();
		System.out.print("StartDate (YYYY-MM-DD): "); sd = sc.nextLine();
		System.out.print("Status: "); st = sc.nextLine();
	}
	
	try (CallableStatement cstmt = conn.prepareCall(
	    "{call sp_InsertRanger(?,?,?)}")) {
		cstmt.setString(1, rid);
		cstmt.setDate(2, Date.valueOf(sd));
		cstmt.setString(3, st);
		cstmt.execute();
	}
	System.out.println("Inserted ranger.");
}
\end{lstlisting}

\begin{figure}[H]
\centering
\includegraphics[width=\textwidth]{screenshots/task5b_doinsertranger.png}
\caption{Screenshot of doInsertRanger method}
\end{figure}

\subsection{doInsertTeam Method}

\begin{lstlisting}[label={lst5b:doinsertteam}]
private static void doInsertTeam() throws SQLException {
	System.out.print("Enter data manually (M) or from file (F): ");
	String choice = sc.nextLine().toUpperCase();
	String fa, leader;
	
	if (choice.equals("F")) {
		System.out.print("Enter file path: ");
		String filePath = sc.nextLine();
		try (BufferedReader br = new BufferedReader(new FileReader(filePath))) {
			fa = br.readLine();
			leader = br.readLine();
		} catch (IOException e) {
			System.out.println("Error reading file: " + e.getMessage());
			return;
		}
	} else {
		System.out.print("FocusArea: "); fa = sc.nextLine();
		System.out.print("Leader RangerID: "); leader = sc.nextLine();
	}
	
	try (CallableStatement cstmt = conn.prepareCall(
	    "{call sp_InsertRangerTeamWithLeader(?,?,?)}")) {
		cstmt.setString(1, fa);
		cstmt.setNull(2, Types.DATE);
		cstmt.setString(3, leader);
		cstmt.execute();
	}
	System.out.println("Inserted team and leader assignment.");
}
\end{lstlisting}

\begin{figure}[H]
\centering
\includegraphics[width=\textwidth]{screenshots/task5b_doinsertteam.png}
\caption{Screenshot of doInsertTeam method}
\end{figure}

\subsection{doInsertDonation Method}

\begin{lstlisting}[label={lst5b:doinsertdonation}]
private static void doInsertDonation() throws SQLException {
	System.out.print("Enter data manually (M) or from file (F): ");
	String choice = sc.nextLine().toUpperCase();
	String did, amt, pt, cn = null, ct = null, lf = null;
	
	if (choice.equals("F")) {
		System.out.print("Enter file path: ");
		String filePath = sc.nextLine();
		try (BufferedReader br = new BufferedReader(new FileReader(filePath))) {
			did = br.readLine();
			amt = br.readLine();
			pt = br.readLine();
			if (pt.equalsIgnoreCase("CHECK")) {
				cn = br.readLine();
			} else {
				ct = br.readLine();
				lf = br.readLine();
			}
		} catch (IOException e) {
			System.out.println("Error reading file: " + e.getMessage());
			return;
		}
	} else {
		System.out.print("DonorID: "); did = sc.nextLine();
		System.out.print("Amount: "); amt = sc.nextLine();
		System.out.print("PaymentType (CHECK/CARD): "); pt = sc.nextLine();
		if (pt.equalsIgnoreCase("CHECK")) {
			System.out.print("CheckNumber: "); cn = sc.nextLine();
		} else {
			System.out.print("CardType: "); ct = sc.nextLine();
			System.out.print("LastFour: "); lf = sc.nextLine();
		}
	}
	
	try (CallableStatement cstmt = conn.prepareCall(
	    "{call sp_InsertDonation(?,?,?,?,?,?,?,?)}")) {
		cstmt.setString(1, did);
		cstmt.setDate(2, new Date(System.currentTimeMillis()));
		cstmt.setBigDecimal(3, new java.math.BigDecimal(amt));
		cstmt.setString(4, null);
		cstmt.setString(5, pt);
		if (pt.equalsIgnoreCase("CHECK")) {
			cstmt.setString(6, cn);
			cstmt.setNull(7, Types.VARCHAR); 
			cstmt.setNull(8, Types.DATE);
		} else {
			cstmt.setNull(6, Types.VARCHAR);
			cstmt.setString(7, ct);
			cstmt.setString(8, lf);
		}
		cstmt.execute();
	}
	System.out.println("Donation inserted.");
}
\end{lstlisting}

\begin{figure}[H]
\centering
\includegraphics[width=\textwidth]{screenshots/task5b_doinsertdonation.png}
\caption{Screenshot of doInsertDonation method}
\end{figure}

\subsection{doInsertResearcher Method}

\begin{lstlisting}[label={lst5b:doinsertresearcher}]
private static void doInsertResearcher() throws SQLException {
	System.out.print("Enter data manually (M) or from file (F): ");
	String choice = sc.nextLine().toUpperCase();
	String rid, field, hd, sal;
	
	if (choice.equals("F")) {
		System.out.print("Enter file path: ");
		String filePath = sc.nextLine();
		try (BufferedReader br = new BufferedReader(new FileReader(filePath))) {
			rid = br.readLine();
			field = br.readLine();
			hd = br.readLine();
			sal = br.readLine();
		} catch (IOException e) {
			System.out.println("Error reading file: " + e.getMessage());
			return;
		}
	} else {
		System.out.print("ResearcherID: "); rid = sc.nextLine();
		System.out.print("Field: "); field = sc.nextLine();
		System.out.print("HireDate (YYYY-MM-DD): "); hd = sc.nextLine();
		System.out.print("Salary: "); sal = sc.nextLine();
	}
	
	try (CallableStatement cstmt = conn.prepareCall(
	    "{call sp_InsertResearcher(?,?,?,?)}")) {
		cstmt.setString(1, rid);
		cstmt.setString(2, field);
		cstmt.setDate(3, Date.valueOf(hd));
		cstmt.setBigDecimal(4, new java.math.BigDecimal(sal));
		cstmt.execute();
	}
	System.out.println("Researcher inserted.");
}
\end{lstlisting}

\begin{figure}[H]
\centering
\includegraphics[width=\textwidth]{screenshots/task5b_doinsertresearcher.png}
\caption{Screenshot of doInsertResearcher method}
\end{figure}

\subsection{doInsertReport Method}

\begin{lstlisting}[label={lst5b:doinsertreport}]
private static void doInsertReport() throws SQLException {
	System.out.print("Enter data manually (M) or from file (F): ");
	String choice = sc.nextLine().toUpperCase();
	String tidStr, rid, sum;
	
	if (choice.equals("F")) {
		System.out.print("Enter file path: ");
		String filePath = sc.nextLine();
		try (BufferedReader br = new BufferedReader(new FileReader(filePath))) {
			tidStr = br.readLine();
			rid = br.readLine();
			sum = br.readLine();
		} catch (IOException e) {
			System.out.println("Error reading file: " + e.getMessage());
			return;
		}
	} else {
		System.out.print("TeamID: "); tidStr = sc.nextLine();
		System.out.print("ResearcherID: "); rid = sc.nextLine();
		System.out.print("Summary: "); sum = sc.nextLine();
	}
	
	int tid = Integer.parseInt(tidStr);
	try (CallableStatement cstmt = conn.prepareCall(
	    "{call sp_InsertReport(?,?,?,?)}")) {
		cstmt.setInt(1, tid); 
		cstmt.setString(2, rid); 
		cstmt.setDate(3, new Date(System.currentTimeMillis())); 
		cstmt.setString(4, sum);
		cstmt.execute();
	}
	System.out.println("Report inserted.");
}
\end{lstlisting}

\begin{figure}[H]
\centering
\includegraphics[width=\textwidth]{screenshots/task5b_doinsertreport.png}
\caption{Screenshot of doInsertReport method}
\end{figure}

\subsection{doInsertProgram Method}

\begin{lstlisting}[label={lst5b:doinsertprogram}]
private static void doInsertProgram() throws SQLException {
	System.out.print("Enter data manually (M) or from file (F): ");
	String choice = sc.nextLine().toUpperCase();
	String pn, name, type, sd, dur;
	
	if (choice.equals("F")) {
		System.out.print("Enter file path: ");
		String filePath = sc.nextLine();
		try (BufferedReader br = new BufferedReader(new FileReader(filePath))) {
			pn = br.readLine();
			name = br.readLine();
			type = br.readLine();
			sd = br.readLine();
			dur = br.readLine();
		} catch (IOException e) {
			System.out.println("Error reading file: " + e.getMessage());
			return;
		}
	} else {
		System.out.print("ParkName: "); pn = sc.nextLine();
		System.out.print("ProgramName: "); name = sc.nextLine();
		System.out.print("Type: "); type = sc.nextLine();
		System.out.print("StartDate (YYYY-MM-DD): "); sd = sc.nextLine();
		System.out.print("Duration hours: "); dur = sc.nextLine();
	}
	
	try (CallableStatement cstmt = conn.prepareCall(
	    "{call sp_InsertProgram(?,?,?,?,?)}")) {
		cstmt.setString(1, pn); 
		cstmt.setString(2, name); 
		cstmt.setString(3, type); 
		cstmt.setDate(4, Date.valueOf(sd)); 
		cstmt.setInt(5, Integer.parseInt(dur));
		cstmt.execute();
	}
	System.out.println("Program inserted.");
}
\end{lstlisting}

\begin{figure}[H]
\centering
\includegraphics[width=\textwidth]{screenshots/task5b_doinsertprogram.png}
\caption{Screenshot of doInsertProgram method}
\end{figure}

\section{Retrieval Methods}

\subsection{doGetEmergencyContacts Method}

\begin{lstlisting}[label={lst5b:dogetemergencycontacts}]
private static void doGetEmergencyContacts() throws SQLException {
	System.out.print("PersonID: "); 
	String pid = sc.nextLine();
	try (CallableStatement cstmt = conn.prepareCall(
	    "{call sp_GetEmergencyContacts(?)}")) {
		cstmt.setString(1, pid);
		try (ResultSet rs = cstmt.executeQuery()) {
			while (rs.next()) 
			    System.out.println(rs.getString(1) + " | " + 
			                       rs.getString(2) + " | " + 
			                       rs.getString(3));
		}
	}
}
\end{lstlisting}

\begin{figure}[H]
\centering
\includegraphics[width=\textwidth]{screenshots/task5b_dogetemergencycontacts.png}
\caption{Screenshot of doGetEmergencyContacts method}
\end{figure}

\subsection{doGetVisitorsByProgram Method}

\begin{lstlisting}[label={lst5b:dogetvisitorsbyprogram}]
private static void doGetVisitorsByProgram() throws SQLException {
	System.out.print("ProgramID: "); 
	String pidStr = sc.nextLine();
	int pid = Integer.parseInt(pidStr);
	try (CallableStatement cstmt = conn.prepareCall(
	    "{call sp_GetVisitorsByProgram(?)}")) {
		cstmt.setInt(1, pid);
		try (ResultSet rs = cstmt.executeQuery()) {
			while (rs.next()) 
			    System.out.println(rs.getString(1) + " | " + 
			                       rs.getString(2) + " | " + 
			                       rs.getString(3));
		}
	}
}
\end{lstlisting}

\begin{figure}[H]
\centering
\includegraphics[width=\textwidth]{screenshots/task5b_dogetvisitorsbyprogram.png}
\caption{Screenshot of doGetVisitorsByProgram method}
\end{figure}

\subsection{doGetProgramsAfterDate Method}

\begin{lstlisting}[label={lst5b:dogetprogramsafterdate}]
private static void doGetProgramsAfterDate() throws SQLException {
	System.out.print("ParkName: "); String pn = sc.nextLine();
	System.out.print("StartAfter (YYYY-MM-DD): "); String sd = sc.nextLine();
	try (CallableStatement cstmt = conn.prepareCall(
	    "{call sp_GetProgramsAfterDate(?,?)}")) {
		cstmt.setString(1, pn); 
		cstmt.setDate(2, Date.valueOf(sd));
		try (ResultSet rs = cstmt.executeQuery()) {
			while (rs.next()) 
			    System.out.println(rs.getInt("ProgramID") + ": " + 
			                       rs.getString("ProgramName") + " - " + 
			                       rs.getDate("StartDate"));
		}
	}
}
\end{lstlisting}

\begin{figure}[H]
\centering
\includegraphics[width=\textwidth]{screenshots/task5b_dogetprogramsafterdate.png}
\caption{Screenshot of doGetProgramsAfterDate method}
\end{figure}

\subsection{doAnonymousDonationStats Method}

\begin{lstlisting}[label={lst5b:doanonymousdonationstats}]
private static void doAnonymousDonationStats() throws SQLException {
	System.out.print("Year: "); int y = Integer.parseInt(sc.nextLine());
	System.out.print("Month (1-12): "); int m = Integer.parseInt(sc.nextLine());
	try (CallableStatement cstmt = conn.prepareCall(
	    "{call sp_AnonymousDonationStatsByMonth(?,?)}")) {
		cstmt.setInt(1, y); 
		cstmt.setInt(2, m);
		try (ResultSet rs = cstmt.executeQuery()) {
			while (rs.next()) 
			    System.out.println(rs.getString("DonorID") + " - Total: " + 
			                       rs.getBigDecimal("TotalAmount") + " Avg: " + 
			                       rs.getBigDecimal("AvgAmount") + " Count: " + 
			                       rs.getInt("NumDonations"));
		}
	}
}
\end{lstlisting}

\begin{figure}[H]
\centering
\includegraphics[width=\textwidth]{screenshots/task5b_doanonymousdonationstats.png}
\caption{Screenshot of doAnonymousDonationStats method}
\end{figure}

\subsection{doGetRangersInTeam Method}

\begin{lstlisting}[label={lst5b:dogetrangersinteam}]
private static void doGetRangersInTeam() throws SQLException {
	System.out.print("TeamID: "); int tid = Integer.parseInt(sc.nextLine());
	try (CallableStatement cstmt = conn.prepareCall(
	    "{call sp_GetRangersInTeam(?)}")) {
		cstmt.setInt(1, tid);
		try (ResultSet rs = cstmt.executeQuery()) {
			while (rs.next()) 
			    System.out.println(rs.getString("RangerID") + " - " + 
			                       rs.getString("FirstName") + " " + 
			                       rs.getString("LastName") + " Years: " + 
			                       rs.getInt("YearsOfService") + " Leader: " + 
			                       rs.getBoolean("IsLeader") + " Cert: " + 
			                       rs.getString("Certification"));
		}
	}
}
\end{lstlisting}

\begin{figure}[H]
\centering
\includegraphics[width=\textwidth]{screenshots/task5b_dogetrangersinteam.png}
\caption{Screenshot of doGetRangersInTeam method}
\end{figure}

\subsection{doGetAllIndividuals Method}

\begin{lstlisting}[label={lst5b:dogetallindividuals}]
private static void doGetAllIndividuals() throws SQLException {
	try (CallableStatement cstmt = conn.prepareCall(
	    "{call sp_GetAllIndividuals}")) {
		try (ResultSet rs = cstmt.executeQuery()) {
			while (rs.next()) 
			    System.out.println(rs.getString("PersonID") + " - " + 
			                       rs.getString("FirstName") + " " + 
			                       rs.getString("LastName") + " Newsletter: " + 
			                       rs.getBoolean("NewsletterSubscribed"));
		}
	}
}
\end{lstlisting}

\begin{figure}[H]
\centering
\includegraphics[width=\textwidth]{screenshots/task5b_dogetallindividuals.png}
\caption{Screenshot of doGetAllIndividuals method}
\end{figure}

\section{Modification Methods}

\subsection{doIncreaseSalary Method}

\begin{lstlisting}[label={lst5b:doincreasesalary}]
private static void doIncreaseSalary() throws SQLException {
	try (CallableStatement cstmt = conn.prepareCall(
	    "{call sp_IncreaseSalary}")) {
		cstmt.execute();
	}
	System.out.println("Salaries updated where applicable.");
}
\end{lstlisting}

\begin{figure}[H]
\centering
\includegraphics[width=\textwidth]{screenshots/task5b_doincreasesalary.png}
\caption{Screenshot of doIncreaseSalary method}
\end{figure}

\subsection{doDeleteInactiveVisitors Method}

\begin{lstlisting}[label={lst5b:dodeleteinactivevisitors}]
private static void doDeleteInactiveVisitors() throws SQLException {
	try (CallableStatement cstmt = conn.prepareCall(
	    "{call sp_DeleteInactiveVisitors}")) {
		cstmt.execute();
	}
	System.out.println("Inactive visitors removed.");
}
\end{lstlisting}

\begin{figure}[H]
\centering
\includegraphics[width=\textwidth]{screenshots/task5b_dodeleteinactivevisitors.png}
\caption{Screenshot of doDeleteInactiveVisitors method}
\end{figure}

\section{Import and Export Methods}

\subsection{doImportTeams Method}

This method reads a CSV file and imports ranger teams.

\begin{lstlisting}[label={lst5b:doimportteams}]
private static void doImportTeams() throws SQLException {
	System.out.print("Enter CSV file path: "); 
	String path = sc.nextLine();
	try (BufferedReader br = new BufferedReader(new FileReader(path))) {
		String line;
		while ((line = br.readLine()) != null) {
			String[] cols = line.split(",");
			if (cols.length < 3) continue;
			CallableStatement cstmt = conn.prepareCall(
			    "{call sp_InsertRangerTeamWithLeader(?,?,?)}");
			cstmt.setString(1, cols[0]);
			if (cols[1].isBlank()) 
			    cstmt.setNull(2, Types.DATE); 
			else 
			    cstmt.setDate(2, Date.valueOf(cols[1]));
			cstmt.setString(3, cols[2]);
			cstmt.execute(); 
			cstmt.close();
		}
	} catch (Exception e) {
		System.out.println("Import error: " + e.getMessage());
	}
	System.out.println("Import finished.");
}
\end{lstlisting}

\begin{figure}[H]
\centering
\includegraphics[width=\textwidth]{screenshots/task5b_doimportteams.png}
\caption{Screenshot of doImportTeams method}
\end{figure}

\subsection{doExportMailingList Method}

This method exports newsletter subscribers to a CSV file.

\begin{lstlisting}[label={lst5b:doexportmailinglist}]
private static void doExportMailingList() throws SQLException {
	System.out.print("Enter output CSV file path: "); 
	String path = sc.nextLine();
	try (java.io.PrintWriter pw = new java.io.PrintWriter(path)) {
		PreparedStatement pst = conn.prepareStatement(
		    "SELECT p.PersonID, p.FirstName, p.LastName, p.Street, " +
		    "p.City, p.State, p.PostalCode FROM Person p " +
		    "WHERE NewsletterSubscribed = 1");
		ResultSet rs = pst.executeQuery();
		pw.println("PersonID,FirstName,LastName,Street,City,State,PostalCode");
		while (rs.next()) {
			pw.printf("%s,%s,%s,%s,%s,%s,%s\n", 
			    rs.getString(1), rs.getString(2), rs.getString(3), 
			    safe(rs.getString(4)), safe(rs.getString(5)), 
			    safe(rs.getString(6)), safe(rs.getString(7)));
		}
		rs.close(); 
		pst.close();
	} catch (Exception e) { 
	    System.out.println("Export error: " + e.getMessage()); 
	}
	System.out.println("Export complete.");
}

private static String safe(String s) {
	return (s == null) ? "" : s;
}
\end{lstlisting}

\begin{figure}[H]
\centering
\includegraphics[width=\textwidth]{screenshots/task5b_doexportmailinglist.png}
\caption{Screenshot of doExportMailingList method}
\end{figure}

\subsection{exportAllTables Method}

This method exports all database tables to CSV files after modifying operations.

\begin{lstlisting}[label={lst5b:exportalltables}]
private static void exportAllTables() {
	try {
		Files.createDirectories(EXPORT_DIR);
		for (String table : TABLES) {
			exportTable(table);
		}
		System.out.println("Table export complete. Files saved to " + 
		                   EXPORT_DIR.toAbsolutePath());
	} catch (Exception e) {
		System.out.println("Export error: " + e.getMessage());
	}
}
\end{lstlisting}

\begin{figure}[H]
\centering
\includegraphics[width=\textwidth]{screenshots/task5b_exportalltables.png}
\caption{Screenshot of exportAllTables method}
\end{figure}

\subsection{exportTable Method}

This helper method exports a single table to CSV format.

\begin{lstlisting}[label={lst5b:exporttable}]
private static void exportTable(String tableName) 
        throws SQLException, IOException {
	String query = "SELECT * FROM dbo." + tableName;
	try (Statement statement = conn.createStatement();
	     ResultSet rs = statement.executeQuery(query)) {
		ResultSetMetaData meta = rs.getMetaData();
		int columnCount = meta.getColumnCount();
		Path csvPath = EXPORT_DIR.resolve(tableName.toLowerCase() + ".csv");
		
		try (Writer writer = new BufferedWriter(
		        new OutputStreamWriter(
		            new FileOutputStream(csvPath.toFile()),
		            StandardCharsets.UTF_8))) {
			// Write header
			for (int i = 1; i <= columnCount; i++) {
				writer.write(escapeCsv(meta.getColumnLabel(i)));
				if (i < columnCount) {
					writer.write(',');
				}
			}
			writer.write(System.lineSeparator());
			
			int rowCount = 0;
			while (rs.next()) {
				rowCount++;
				for (int i = 1; i <= columnCount; i++) {
					Object value = rs.getObject(i);
					writer.write(escapeCsv(formatValue(value)));
					if (i < columnCount) {
						writer.write(',');
					}
				}
				writer.write(System.lineSeparator());
			}
			writer.flush();
			System.out.printf("Exported %-22s -> %s (%d rows)%n",
			    tableName, csvPath.toString(), rowCount);
		}
	}
}
\end{lstlisting}

\begin{figure}[H]
\centering
\includegraphics[width=\textwidth]{screenshots/task5b_exporttable.png}
\caption{Screenshot of exportTable method}
\end{figure}

\subsection{Helper Methods for CSV Export}

\begin{lstlisting}[label={lst5b:csvhelpers}]
private static String formatValue(Object value) {
	if (value == null) {
		return "";
	}
	if (value instanceof Date) {
		return value.toString();
	}
	if (value instanceof java.sql.Timestamp) {
		return TIMESTAMP_FORMAT.format(new java.util.Date(
		    ((java.sql.Timestamp) value).getTime()));
	}
	if (value instanceof java.sql.Time) {
		return value.toString();
	}
	return value.toString();
}

private static String escapeCsv(String input) {
	if (input == null) {
		return "";
	}
	boolean needQuotes = input.contains(",")
	    || input.contains("\"")
	    || input.contains("\n")
	    || input.contains("\r");
	String result = input.replace("\"", "\"\"");
	if (needQuotes) {
		result = "\"" + result + "\"";
	}
	return result;
}
\end{lstlisting}

\begin{figure}[H]
\centering
\includegraphics[width=\textwidth]{screenshots/task5b_csvhelpers.png}
\caption{Screenshot of CSV helper methods}
\end{figure}

\section{Compilation and Execution}

The Java application was successfully compiled and tested with all stored procedures.

\begin{figure}[H]
\centering
\includegraphics[width=0.9\linewidth]{screenshots/java_proof.png}
\caption{Screenshot of successful compilation}
\end{figure}

\section{Summary}

The Java application provides a complete interface to the NPSS database with the following capabilities:

\begin{itemize}
    \item \textbf{Insert Operations:} All 7 insert stored procedures (visitors, rangers, teams, donations, researchers, reports, programs)
    \item \textbf{Query Operations:} All 6 retrieval stored procedures (emergency contacts, visitors by program, programs after date, anonymous donation stats, rangers in team, all individuals)
    \item \textbf{Modification Operations:} 2 procedures that modify data (salary increase, delete inactive visitors)
    \item \textbf{Import/Export:} CSV import for teams and export for mailing lists
    \item \textbf{Automatic Table Export:} After each modifying operation, all tables are exported to CSV for documentation
    \item \textbf{Error Handling:} Comprehensive try-catch blocks and user-friendly error messages
    \item \textbf{Flexible Input:} Manual entry or file-based input for all insert operations
\end{itemize}

The application demonstrates proper JDBC usage including:
\begin{itemize}
    \item Connection management with properties file
    \item CallableStatement for stored procedure execution
    \item PreparedStatement for direct queries
    \item ResultSet processing with metadata inspection
    \item Proper resource cleanup with try-with-resources
    \item CSV parsing and generation with proper escaping
\end{itemize}